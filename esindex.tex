%
% Copyright (C) 1998-2019 Javier Bezos http://www.texnia.com
%
% This file may be distributed and/or modified under the conditions of
% the MIT License. A version can be found at the end of this file.
%
% Repository: https://github.com/jbezos/esindex
%

\documentclass[a4paper]{ltxguide}

\usepackage[english,spanish]{babel}
\spanishdatedel
\usepackage[utf8]{inputenc}
\usepackage[T1]{fontenc}

\title{\textsf{esindex}\\\large 1.6\quad 2019-09-30}

\author{Javier Bezos}

\raggedright
\parskip=.8ex
\advance\oddsidemargin-.7cm
\advance\textwidth1.8cm
\addtolength{\textheight}{3.5cm}
\addtolength{\topmargin}{-2cm}

\usepackage{xcolor,bera}

\definecolor{notes}{rgb}{.75, .3, .3}%

\makeatletter
\def\@begintheorem#1#2{%
  \list{}{}%
  \global\advance\@listdepth\m@ne
  \item[{\sffamily\bfseries\color{notes}\MakeUppercase{#1}}]}%
\makeatother
\newtheorem{warning}{Warning}
\newtheorem{note}{Note}
\newtheorem{example}{Example}

\begin{document}

\vspace*{1cm}
{\fontsize{48}{48}\selectfont \color{notes}{esindex}\par}
{\LARGE Generating automatically sort keys for \textit{MakeIndex}
with \LaTeX\par}
\vspace*{1ex}
Version 1.6 (2019-09-30)\par
Javier Bezos (\texttt{http://www.texnia.com})

\vspace*{6ex}

{\small\itshape Please, report any issues you find in
  \texttt{https://github.com/jbezos/esindex/issues}.\par}

This package defines the command \verb|\esindex|, which adds sort keys
to the index entries for you. Although originally a specifically Spanish
tool (so part of the documentation is in Spanish), the idea behind this
package may be applied to other languages, and tools to adapt it are
provided. They are explained below, section 2.

So,
\begin{verbatim}
\esindex{cañón}
\end{verbatim}
is equivalent to
\begin{verbatim}
\index{can^^ffon@cañón}
\end{verbatim}

\verb|\index| is not touched at all (except if requested with the package
option \verb|replaceindex|).

As you can see, the package generates the sort key within \TeX{}
itself, which has a number of advantages (and some disadvantages, of
course).

Version 1.5 provides tools for other languages and adds support for
\textsf{luatex} and \textsf{xetex}.

Version 1.6 adds a package option -- with \verb|babel| the `actual' value
(ie, the text to be printed in the index) is wrapped with
\verb|\foreignlanguage| if the current language is not the main one
(this replacement comes after \verb|\esindexactual|). Actually, the
macro is \verb|\esindexlanguage|, which by default is `let' to the
\textsf{babel} macro.

\section{Spanish}

Este paquete ha sido diseñado para facilitar la escritura de índices
correctamente alfabetizados en castellano. Su principal orden es
\verb|\esindex|, que convierte a una forma adecuada su argumento. Así
por ejemplo,
\begin{verbatim}
\esindex{cañón}
\end{verbatim}
equivale a
\begin{verbatim}
\index{can^^ffon@cañón}
\end{verbatim}
No es necesario usar \textsf{babel} salvo, lógicamente, si los acentos
están escritos en forma de abreviaciones (\verb|'a|, \verb|'e|, etc.)
en lugar de con los caracteres reales. En este último caso, el paquete
utiliza ciertas órdenes internas de \textsf{babel}, por lo que no puedo
garantizar su funcionamiento correcto con versiones distintas a las 3.6
a 3.27. En caso de que \textsf{esindex} sea incompatible con futuras
versiones de \textsf{babel} (lo cual no es realmente probable)
intentaré adaptarlo en el menor tiempo posible.

Salvo el carácter \verb|actual| (normalmente \verb|@|) se pueden usar
todos los caracteres especiales de \textit{MakeIndex}. Se pueden
aplicar convenciones diferentes a las normales, pero en este caso hacen
falta ajustes adicionales en caso de que los modificados sean
\verb|actual|, \verb|encap|, \verb|level| o \verb|quote|. En ese caso
basta con indicar los caracteres que hay que usar como opciones de
paquete. Por ejemplo, si para \verb|quote| decidimos usar \verb|$| en
nuestro archivo \verb|.ist| particular, tendríamos que llamar al
paquete del siguiente modo:
\begin{verbatim}
\usepackage[quote=$]{esindex}
\end{verbatim}

Es importante observar que, a diferencia de la opción para alemán de
\textit{MakeIndex}, el uso de \verb|"| en abreviaciones como \verb|"u|
es completamente legítimo, ya que el paquete reconoce tal combinación y
la trata aparte. Lo mismo vale para \verb|'| o \verb|~| en caso de que
se usaran como carácter especial. Es decir
\begin{verbatim}
\esindex{{"!`}Cig"ue'nas{"!}|textbf}
\end{verbatim}
equivale a
\begin{verbatim}
\index{{"!`}Ciguen^^ffas{"!}@{"!`}Cig\"ue\~nas{"!}|textbf}
\end{verbatim}

Sin embargo, el uso del carácter \verb|quote| ante \verb|encap| o
\verb|level| no se detecta a menos que el grupo esté encerrado entre
llaves. Por ejemplo, en lugar de \verb/\esindex{Pleca: "|}/ debe
escribirse \verb/\esindex{Pleca: {"|}}/. (En realidad en este caso
podría haberse usado \verb|\index|. Es tan sólo un ejemplo.)

Aunque el hecho de que \verb|@| no se pueda usar en \verb|\esindex|
hace que todavía algunas entradas se tengan que hacer a mano, la mayor
parte del trabajo se ve considerablemente simplificado.

Finalmente, hay que señalar que con este paquete no se crea en el
índice una entrada propia para la palabras que empiezan por eñe, sino
que tan sólo se añaden al final de la ene. En el rarísimo caso de que
hubiera palabras que empiezan por eñe habría que modificar el archivo
\verb|.ind| a mano o bien redefinir de algún modo las entradas
generadas.

La versión 1.3 elimina una incompatibilidad con recientes versiones de
\LaTeX{} y añade nuevas funciones:
\begin{itemize}
\item Opción de paquete \verb|ignorespaces|: al formar la clave de 
ordenación se suprimen los espacios, de forma que:
$$\mbox{adentro} < \mbox{a donde} = \mbox{adonde}.$$
\item Opción de paquete \verb|replaceindex|: el comportamiento de
\verb|\index| se reemplaza por el de \verb|\esindex|, aunque en este
caso no es posible introducir entradas que no se adapten a lo
requerido por \verb|\esindex|.
\item La orden \verb|\ignorewords| da
una lista de palabras separadas por comas que no se cuentan en la
ordenación.  Por ejemplo, con \verb|\ignorewords{de}| tendríamos:
$$\mbox{pino albar} < \mbox{pino laricio} < \mbox{pino de montaña}.$$
Distingue la caja, por lo que las formas con mayúsculas hay que darlas
explícitamente, si hicieran falta.
\end{itemize}

Algunas funciones adicionales son:
\begin{itemize}
\item La lista de tókenes \verb|\everyesindex| permite dar definiciones
locales para establecer el comportamiento de otras órdenes. Por
ejemplo:
\begin{verbatim}
\everyesindex{\renewcommand\emph[1]{#1}}
\end{verbatim}
elimina de la clave esa orden. Con:
\begin{verbatim}
\everyesindex{\renewcommand\emph[1]{#1'}}
\end{verbatim}
la entrada en cursiva iría detrás de la redonda, si la hubiera. Es una
técnica que se puede emplear en otros casos para reajustar el orden de
entradas idénticas.

\item La orden \verb|\esindexsort| permite predefinir claves asociadas
a entradas concretas, para ajustar su ordenación (lo que normalmente se
consigue añadiendo texto adicional para que \textsf{makeindex} lo tenga
en cuenta). Estas correspondencias deben darse antes de la aparición
del primer \verb|\esindex| con ese término, y las claves se procesan
posteriormente con \verb|ignorespaces|, \verb|\ignorewords| y
\verb|\everyesindex|, si están activadas. Por ejemplo:
\begin{verbatim}
\esindexsort{adonde}{adonde'1}
\esindexsort{adónde}{adonde'2}
\esindexsort{a donde}{a donde'7}
\esindexsort{a dónde}{a donde'8}
\end{verbatim}
daría el orden \emph{adonde, adónde, a donde, a dónde}, con
\verb|ignorespaces|, o bien \emph{a donde, a dónde,} [probablemente
otros términos], \emph{adonde, adónde}, sin \verb|ignorespaces|.
\end{itemize}

\section{Other languages}

First, you very likely want to ignore de Spanish specific settings with
the package option \verb|nospanish|. What \verb|\esindex| does is the
following:

\begin{enumerate}
\item Replacements set by \verb|\esindexsort|. See an example above.

\item \verb|\esindexreplace| in \verb|\everyesindex|. See an example
below.

\item Replacements by \LaTeX{} (protected) expansion, including
redefinitions in \verb|\everyesindex| and, in Spanish, \verb|\`|,
\verb|\"| and \verb|\~|. You may force protected expansion at any point
inside \verb|\everyesindex| with \verb|\esindexexpandkey|. In addition,
the sort key is stored in \verb|\esindexkey|, which can be manipulated
directly.

\item Removal of words listed in \verb|\ignoredwords| (maybe not the
logical place, but real life is not always logical). For example, with
\verb|\ignorewords{de,la}| the words “de” and “la” (preceded and
followed by spaces) are removed from the key.

\item Removal of spaces if \verb|ignorespaces|. Very often this is
similar to the \verb|-l| option of \textit{MakeIndex}.

\item Replacement of `actual' as set by \verb|\esindexactual|, but
still based on the original \verb|\esindex| argument, without changes.
For example, with:
\begin{verbatim}
\esindexactual{Felipe II}{Felipe II, \textit{rey de España}}
\end{verbatim}
just write \verb|\esindex{Felipe II}| (as many times as you want), and
the entry will show “Felipe II, \textit{rey de España}” (the sort key
is still based on “Felipe II”, of course).

\end{enumerate}

Do you find it chaotic? Well, you are right. After all this package was
for Spanish indexes with some readjustments for it to be adapted to
other languages. A general solution deserves another package. Feel free
to create one based on this package (MIT license), if you like.

Here is an example of an \verb|\everyesindex|. Like other
\verb|\every...|'s, it is a token register:
\begin{verbatim}
\everyesindex{%
  \renewcommand\"[1]{#1e}%
  \renewcommand\textit[1]{#1}%
  \esindexexpandkey
  \esindexreplace{ä}{ae}%
  \esindexreplace{ü}{ue}%
  \esindexreplace{ö}{oe}}
\end{verbatim}

What it does is:
\begin{enumerate}
\item Redefines \verb|\textit| so that it is removed to build the key.
With \verb|\everyesindex{\renewcommand\textit[1]{#1'}}| italics would
be sorted after upright.

\item Redefines \verb|\"|. Of course, this only works correctly if used
for this precise expansion.

\item Applies the previous changes with a protected expansion.

\item Make a direct replacement of some characters.

\end{enumerate}
Note there is a pattern: first flatten the key by removing macros, and
then the resulting unmarked text is processed to replace some chars.

Remember these changes are not shown -- they are used by
\textit{MakeIndex} to sort the entries. Note also replacements can be
made language dependent with the appropriate test (eg, with
\textsf{iflang}).

As a convenience tool, \verb|\esindexlastchar| is \verb|^^ff| or
\verb|^^^^ffff|, depending on the engine.

You can play with \textsf{lua} inside \verb|\everyesindex|, too. For
example, to convert at once things like \textit{2,3-dimetilfenol} to
\textit{dimetilfenol}:
\begin{verbatim}
\everyesindex{
  \directlua{
    token.set_macro('esindexkey',
                    string.gsub([[\esindexkey]], '^[-0-9,]+', ''))
  }}
\end{verbatim}

It is just the concept (for example, it fails if the key contains
\verb|]]|).

\section{Final remarks}

Indexing has many subtleties. Recommended readings are:
\begin{itemize}
  \item \textit{Guidelines for Alphabetical Arrangement of Letters and
  Sorting of Numerals and Other Symbols},
  \verb|https://www.niso.org/sites/default/files/2017-08/tr03.pdf|
  \item \textit{Guidelines for Indexes and Related Information Retrieval
  Devices},
  \verb|https://www.niso.org/sites/default/files/2017-08/tr02.pdf|
  \item \textit{Unicode Collation Algorithm},
  \verb|https://unicode.org/reports/tr10/|,
  \item \textit{Unicode Locale Data Markup Language (LDML)}, part 5,
  \textit{Collation},
  \verb|http://www.unicode.org/reports/tr35/tr35-collation.html|.
\end{itemize}

There are at least 3 tasks, from \LaTeX{} to a suitable key (eg,
\verb|$\delta$| to \verb|delta| or whatever you want), collating these
keys, and generate the formated index in a form \LaTeX{}
understands.\footnote{There can be also sub-tasks. For example, the
last task is best split in two sub-tasks, namely, first generate a
sorted list of terms, and then format them, but unfortunately, most
\texttt{ist} styles output a formatted list. We also need to format and
collapse locators and so on.} With \textsf{makeindex} these 3 steps are
intermingled, and so does \textsf{esindex}. This is another reason why
a general solution deserves a completely new package. However, with
some patience you can do many useful things with \textsf{esindex}, even
sorting words in non-Latin scripts (at least, with the most basic
rules).

\end{document}

MIT License
-----------

Permission is hereby granted, free of charge, to any person obtaining a
copy of this software and associated documentation files (the
"Software"), to deal in the Software without restriction, including
without limitation the rights to use, copy, modify, merge, publish,
distribute, sublicense, and/or sell copies of the Software, and to
permit persons to whom the Software is furnished to do so, subject to
the following conditions:

The above copyright notice and this permission notice shall be included
in all copies or substantial portions of the Software.

THE SOFTWARE IS PROVIDED "AS IS", WITHOUT WARRANTY OF ANY KIND, EXPRESS
OR IMPLIED, INCLUDING BUT NOT LIMITED TO THE WARRANTIES OF
MERCHANTABILITY, FITNESS FOR A PARTICULAR PURPOSE AND NONINFRINGEMENT.
IN NO EVENT SHALL THE AUTHORS OR COPYRIGHT HOLDERS BE LIABLE FOR ANY
CLAIM, DAMAGES OR OTHER LIABILITY, WHETHER IN AN ACTION OF CONTRACT,
TORT OR OTHERWISE, ARISING FROM, OUT OF OR IN CONNECTION WITH THE
SOFTWARE OR THE USE OR OTHER DEALINGS IN THE SOFTWARE.
